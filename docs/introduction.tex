\section{Introduction}

\subsection{Background}

There is broad agreement among commercial, academic, and government
leaders about the remarkable potential of Big Data to spark
innovation, fuel commerce, and drive progress. Big Data is the common
term used to describe the deluge of data in today's networked,
digitized, sensor-laden, and information-driven world. The
availability of vast data resources carries the potential to answer
questions previously out of reach, including the following:

\begin{itemize}
\item How can a potential pandemic reliably be detected early enough
  to intervene?
\item Can new materials with advanced properties be predicted before
  these materials have ever been synthesized?
\item How can the current advantage of the attacker over the defender
  in guarding against cyber-security threats be reversed?
\end{itemize}

There is also broad agreement on the ability of Big Data to overwhelm
traditional approaches. The growth rates for data volumes, speeds, and
complexity are outpacing scientific and technological advances in data
analytics, management, transport, and data user spheres.  Despite
widespread agreement on the inherent opportunities and current
limitations of Big Data, a lack of consensus on some important
fundamental questions continues to confuse potential users and stymie
progress. These questions include the following:

\begin{itemize}

\item	How is Big Data defined?

\item	What attributes define Big Data solutions? 

\item	What is new in Big Data?

\item What is the difference between Big Data and bigger data that has
been collected for years?

\item	How is Big Data different from traditional data environments and related applications? 

\item	What are the essential characteristics of Big Data environments? 

\item	How do these environments integrate with currently deployed architectures? 

\item What are the central scientific, technological, and
standardization challenges that need to be addressed to accelerate the
deployment of robust, secure Big Data solutions?

\end{itemize}

Within this context, on March 29, 2012, the White House announced the
Big Data Research and Development Initiative. The initiative's goals
include helping to accelerate the pace of discovery in science and
engineering, strengthening national security, and transforming
teaching and learning by improving analysts' ability to extract
knowledge and insights from large and complex collections of digital
data.

Six federal departments and their agencies announced more than \$200
million in commitments spread across more than 80 projects, which aim
to significantly improve the tools and techniques needed to access,
organize, and draw conclusions from huge volumes of digital data. The
initiative also challenged industry, research universities, and
nonprofits to join with the federal government to make the most of the
opportunities created by Big Data.

Motivated by the White House initiative and public suggestions, the
National Institute of Standards and Technology (NIST) has accepted the
challenge to stimulate collaboration among industry professionals to
further the secure and effective adoption of Big Data. As one result
of NIST's Cloud and Big Data Forum held on January 15–17, 2013, there
was strong encouragement for NIST to create a public working group for
the development of a Big Data Standards Roadmap. Forum participants
noted that this roadmap should define and prioritize Big Data
requirements, including interoperability, portability, reusability,
extensibility, data usage, analytics, and technology infrastructure.
In doing so, the roadmap would accelerate the adoption of the most
secure and effective Big Data techniques and technology.

On June 19, 2013, the NIST Big Data Public Working Group (NBD-PWG) was
launched with extensive participation by industry, academia, and
government from across the nation. The scope of the NBD-PWG involves
forming a community of interests from all sectors—including industry,
academia, and government—with the goal of developing consensus on
definitions, taxonomies, secure reference architectures, security and
privacy, and—from these—a standards roadmap. Such a consensus would
create a vendor-neutral, technology- and infrastructure-independent
framework that would enable Big Data stakeholders to identify and use
the best analytics tools for their processing and visualization
requirements on the most suitable computing platform and cluster,
while also allowing added value from Big Data service providers.

The NIST Big Data Interoperability Framework (NBDIF) will be released
in three versions, which correspond to the three stages of the NBD-PWG
work. The three stages aim to achieve the following with respect to
the NIST Big Data Reference Architecture (NBDRA).

\begin{itemize}

\item Stage 1: Identify the high-level Big Data reference architecture
key components, which are technology, infrastructure, and vendor
agnostic.

\item Stage 2: Define general interfaces between the NBDRA components.

\item Stage 3: Validate the NBDRA by building Big Data general
applications through the general interfaces.
\end{itemize}

On September 16, 2015, seven NBDIF Version 1 volumes were published
(\url{http://bigdatawg.nist.gov/V1_output_docs.php}), each of which
addresses a specific key topic, resulting from the work of the
NBD-PWG. The seven volumes are as follows:

\begin{itemize}
\item	Volume 1, Definitions
\item	Volume 2, Taxonomies 
\item	Volume 3, Use Cases and General Requirements
\item	Volume 4, Security and Privacy 
\item	Volume 5, Architectures White Paper Survey
\item	Volume 6, Reference Architecture
\item	Volume 7, Standards Roadmap
\end{itemize}

Currently, the NBD-PWG is working on Stage 2 with the goals to enhance the Version 1 content, define general interfaces between the NBDRA components by aggregating low-level interactions into high-level general interfaces, and demonstrate how the NBDRA can be used. As a result of the Stage 2 work, the following two additional NBDIF volumes have been identified.

\begin{itemize}
\item	Volume 8, Reference Architecture Interfaces

\item	Volume 9, Adoption and Modernization
\end{itemize}

Version 2 of the NBDIF volumes, resulting from Stage 2 work, can be downloaded from the NBD-PWG website (\url{https://bigdatawg.nist.gov/V2_output_docs.php}). Potential areas of future work for each volume during Stage 3 are highlighted in Section 1.5 of each volume. The current effort documented in this volume reflects concepts developed within the rapidly evolving field of Big Data.




